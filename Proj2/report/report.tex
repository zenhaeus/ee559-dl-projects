\documentclass[10pt,conference,compsocconf]{IEEEtran}

\usepackage{hyperref}
\usepackage{graphicx}	% For figure environment
\usepackage{subcaption}
\usepackage{amsmath}
\usepackage{bm}
\usepackage{tikz}
\usepackage{tikz-qtree}
\usepackage{amsmath}
\usepackage{amssymb}
\usepackage{amsfonts}
\usepackage{amsthm}
\usepackage{mathrsfs}
\usepackage{cleveref}
\usepackage{xcolor}
\usetikzlibrary{trees,calc,arrows.meta,positioning,decorations.pathreplacing,bending}


\begin{document}
\title{Deep Learning - Miniproject 2}

\author{
  Olesya Altunina (285467), Mauro Pfister (235440), Joey Zenhaeusern (226652)
}

\maketitle

\begin{abstract}

\end{abstract}

\section{Introduction}

\section{Framework implementation}
In order to implement the mini deep-learning framework we chose a structure based on the suggestion provided in the project description. All modules are implemented as classes which inherit from a parent class \texttt{Module}. They need to override the \texttt{forward()} and \texttt{backward()} methods.

To ensure the correctness of our implementations we wrote a series of unit tests which compare the functionalities of our code to the corresponding PyTorch functionalities.

\section{Conclusion}

\bibliographystyle{IEEEtran}
\bibliography{literature}

% reference to Bishop for Newton's method

\end{document}
